\documentclass{article}

\usepackage{authblk}
\setlength{\textheight}{655pt} % = 592pt
\setlength{\textwidth}{370pt} % = 345pt
\setlength{\topmargin}{0pt} % = 20pt
\setlength{\voffset}{-35pt} % = 0pt
\setlength{\headsep}{20pt} %= 25pt

\usepackage{fullpage}
\usepackage[curve]{xypic}
 \usepackage{amssymb,amsmath, url,proof,bussproofs} 
\usepackage[usenames,dvipsnames]{color}

\newcommand{\noo}[1]{}
\newcommand{\ptrue}{\sf S}
\newcommand{\pp}{\mathcal P}
\newcommand{\defdby}{\by={def}}  
\newcommand{\naboComp}[1]{{\cal E}(#1)}
\newcommand{\CAP}{\bigcap}
\newcommand{\ungr}[1]{\underline{#1}}
\newcommand{\conc}[1]{} 
\newcommand{\objects}{{\cal O}}
\newcommand{\facts}{{\cal F}}
\newcommand{\states}{{\cal S}}
\newcommand{\powers}{{\cal P}}
\newcommand{\disp}{{\cal D}}
\newcommand{\cause}{{\cal C}}
\newcommand{\dispcomb}{{\pi}}
\newcommand{\irr}{{\cal I}}

\bibliographystyle{apalike}

\hyphenation{}
\newcommand{\rag}[1]{\textbf{\textcolor{magenta}{(R: #1)}}}
\newcommand{\sjur}[1]{\textbf{\textcolor{blue}{(S: #1)}}}
\newcommand{\hannah}[1]{\textbf{\textcolor{OliveGreen}{(H: #1)}}}

\begin{document}

\title{Contemplating counterfactuals: On the connection between agency and metaphysical possibility}
\author[1]{Sjur K. Dyrkolbotn \thanks{s.k.dyrkolbotn@durham.ac.uk}}
\author[2]{Ragnhild H. Jordahl \thanks{ragnhild.jordahl@gmail.com}}
\author[3]{Hannah A. Hansen \thanks{hannaha.hansen@gmail.com}}
\affil[1]{Durham Law School, Durham University, UK}
\affil[2]{Department of Philosophy, University of Bergen, Norway}
\affil[3]{Department of Information Science and Media Studies, University of Bergen, Norway}

\date{}
\maketitle

\section*{Extended Abstract}

We consider the connection between the metaphysics of modality and the philosophy of agency and free will.
In particular, we focus on how the relationship between possibility and agency is captured in formal logics for  reasoning about multi-agent systems in the context of artificial intelligence research. We argue that philosophical insights can be gained from looking to these formalisms, and that they tend to come with implicit philosophical assumptions that we must consider if we are to understand their broader meaning.

Social structures that have been designed with the aid of formal tools, within which social interaction involves both real and artifcial agents, have increasingly been included as notions of social reality. \footnote{The growing importance of the social web over the last 10-15 years serves as an obvious example of this development.} It seems clear that philosophical assessments are especially relevant when formal logics are examined in the context of social reality. In addition, philosophy may offer interesting directions to pursue when further developing the formal apparatus.

In the first part of this paper, we argue for the philosophical claim that the connection between metaphysical modality and agency is significant and should be taken into account in order to arrive at a proper understanding of either of these notions. We observe, in particular, that agency appears to feature crucially in important metaphysical arguments concerning possibility, while metaphysical possibility seem to be at play in important arguments concerning agency. 

In the second part of the paper, we consider and compare logical approaches to modeling agency and logical models used to elaborate on metaphysical theories of modality. We focus attention on branching time temporal logics, particularly variants of \emph{alternating time temporal logic} (ATL) \cite{atl}, and we relate these to the recently proposed \emph{dispositional} account of modality, see \cite{dispmod,MwPw}. The dispositional account makes the connection between possibility, causation and agency clearer at the philosophical level, so providing a formal interpretation of this theory seems like a particularly interesting research challenge.

It has been \hannah{backed up} by formalisms and semi-formalisms that bear close resemblance to many logics considered in the theory of multi-agent systems and in the philosophy of agency, see \cite{powmod,PhDpos}. We give a brief description and overview of the frameworks that have been developed, and we suggest that the relationships between proposals emerging from different strands of thought should be considered further. It seems that by making the connection between metaphysical modality and agency, we can shed new light on a number of well-known issues, both from philosophy and the theory of multi-agent systems.

In the remainder of the extended abstract, we elaborate briefly on the main points of the paper.

\subsection*{Agency in the metaphysics of possibility}

One of the main controversies in contemporary work on metaphysical modality arises from the tension between the theories of Lewis and Kripke respectively \cite{KripkeNN,KripkeIN,LewisPW,LewisCPB}. Both Lewis and Kripke build on the account given by Leibniz \cite{Theodicy}, who held that something is possible if and only it is true in some possible world, and necessary if and only it is true in all of them.

Lewis' theory relies on an ontology which posits the existence of concretely existing possible worlds, completely separated from our own. On the contrary, Kripke's theory is based on an \emph{actualistic} understanding of possible worlds; what actually exists is taken to be that which is part of our world, and all that is possible must, in principle, originate from this actuality.

It is commonly accepted that a powerful argument can be made against Lewis' theory by considering \emph{identity} and \emph{de re} modal claims, that is, claims about what is possible for a particular existing object, the identity of which we know in the actual world. How can it be, for instance, that something which is possible for \emph{me} is witnessed by the existence of some other world, all the while I myself am part of this one? Lewis answers by saying that what is possible for me is witnessed by something which obtains in some possible world for someone who is not me, but is very much like me, namely my \emph{counterpart} \cite{LewisCPB}. 

This answer is held by many to be an affront to our intuitive understanding of modality. In a famous thought experiment \cite{KripkeNN}, Kripke makes this point by considering the possibility that Humphrey won the 1968 US presidential election. Why exactly would Humphrey care if someone very much like him won the election? Surely, when contemplating the possibility of victory, Humphrey is thinking about \emph{himself}?

Kripke's argument, and the question of identity across possible worlds more generally, seems to owe much of its significance from considerations rooted in agency. It seems to be the case that the relevant form of modal agency, involving agents contemplating on the possible, is present in the Humphrey thought experiment. Additionally, it appears to be true that whenever a modal claim regarding objects appear, this is due to some agent engaging in modal reflection that involves these objects.

We will not argue about the exact extent to which modal agency of this kind is a relevant factor when attempting to formulate a philosophical theory of possibility. \footnote{Certainly, we are not proposing that modal agency subsumes or is constitutive of metaphysical possibility; that would involve excluding many possibilities that are often included in a metaphysical account, such as the possibility of a world with no agents (some may want their metaphysical theory to exclude this, but we prefer to remain agnostic about it). We are not, in particular, suggesting any kind of fictionalism about metaphysical possibilities, and the point we are making is not subsumed by previous work in this vein, as that of \cite{ficr,ficrfix}. While agency should also be considered by such theories, their primary concern is with how possible worlds are to be made sense of, and how they come to be. This is not our topic; our argument is that \emph{regardless} of what possible worlds are, it appears that how we \emph{interact} 
with them in our social lives is relevant, also to the formulation of an appropriate metaphysical theory of possibility.} Rather, our point is that quite some relevance is already attributed to it in existing work, and that metaphysical theories therefore need to address modal agency more explicitly. It also seems clear that when doing so, due note should be taken of the fact that thinking about modality is invariably embedded in structures that are present in physical and social reality.

\subsection*{The dispositional account of metaphysical possibility}

On this dispositional account, we see what is metaphyssically possible as determined by dispositions found \emph{in the actual world}. This means that it is both an actualistic and a realistic explanation of modality; we restrict ourselves to the content of this world, and we describe modality as something that is \emph{really present} in our world (i.e. not a phenomenon restricted to the way we tend to use our language for example). To say that something is possible means that there is some actual disposition present in the world, for which this possibility --- this possible state of affairs --- is its manifestation. The manifestations are what characterizes and individuates the dispositions, and the set of these actually existing dispositions are then seen as what determines what is metaphysically possible, and not the other way around. To ground possibility in this way, in the actual world, makes is possible to avoid having to take possible worlds (real or metaphorical) as some sort of primitive notion. The possible worlds can then be viewed as a powerful tool in the modal logics, but according to this view it is not needed, and it is not the right starting point for the explanation of the metaphysical modalities. 
\\
\\
It is important to emphasize that the disponitions mentioned are present in the different objects (or agents) in our world here and now, it is not something that will come about in the future. The \emph{manifestations} of the dispositions are however something that might or might not come about. This means that not all dispositions will be manifested, in fact objects will most likely have dispositions that might never come about. Think of the glass that has the dispositional property of being fragile --- this means that the glass will break if struck with sufficient force, but this disposition to break might very well never come about. Even if the dispositions are never manifested, the existence of dispositional properties is enough to account for the possibility that the glass \emph{might} break or that it \emph{could have been} broken already. 
\\
\\
Powers is a term used both in the philosophy of causation, often more or less as a
synonym for dispositions [Mumford and Anjum, 2011b], but also in the philosophy of
agency, where it has a different, but related, meaning [van Inwagen, 1983]. Roughly
speaking, a power can be seen as a disposition involving agency by way of pointing to
an ability that an agent has to bring about an outcome. In the example above, one
might say of the glass that it is disposed to break, but one might also say of me that
I have the power to break it. It seems wrong, however, to say that I am disposed to
do so, simply because I can.
The distinction seems like it could be very useful when attempting to account for
possibility in terms of disposition. If I claim "it is possible for me to break the glass",
it seems that the disposition of the glass to break if I hit it is no longer a sufficient
truthmaker. What if, for instance, I do not exist, or I am necessarily prevented from
hitting the glass for some other reason? In this case, it seems natural to also make
reference to my power to hit the glass as the truthmaker for the modal claim, not only the dispositional fact that it might break if I do so. Some have suggested that
powers is the fundamental of term when it comes to modality, although this involves,
typically, a notion of power which more or less seems to coincide with and/or subsume
the notion of disposition. I refer to [Jacobs, 2010] for a suggestion in this vein, but also
remark that recent work done on so called "branching possibility" seems to involve a
notion of power that is more directly connected to agency, and also employs logical
tools intimately related to multi-agent formalisms, see [Muller, 2012].
In fact, I suspect that the notion of power could prove especially interesting in the
context of modality when it is not taken to subsume dispositions, but used instead
to point towards an explicit metaphysics of modal agency.

\subsection*{Metaphysical possibility in the theory of agency and free will}

Metaphysical possibility appears to play an unacknowledged role in relation to philosophical debates concerning agency, and to illustrate this, we will focus particularly on the classical debate about whether or not free will is compatible with determinism. We observe, in particular, that arguments either way tend to involve implicit commitments to a particular interpretation of what is meant by saying that something is possible. It is often argued, for instance, that an agent cannot have free will if determinism is true because determinism means that it is not possible for him to act differently than he is already determined to act. This has been challenged by many authors, however, and it seems that many of these challenges rely on the idea, which might not be explicitly stated, that the sense of possibility relevant for making a judgment about free will is not the physical possibility relevant to the question of determinism, but some other notion of possibility, for which the truth of determinism would 
not preclude the existence of alternative actions  for an agent.

Towards a simple argument along these lines, one could begin by pointing out that even if the future of the actual physical world is completely determined, there might be another, metaphysically possible world, identical to the actual one except for the fact that the agent \emph{did} act differently. This might then be taken as a witness to the fact that \emph{if} he had wanted to act differently in the actual world, he \emph{could} have done so, and so has free will, even in the presence of determinism. In light of current theories about metaphysical possibility, however, this argument appears weak; if determinism is true, then the fact that an agent could have acted differently if he had wanted to appears vacuous; after all, it is not possible, in the actual world, that he could have wanted this.

But the simple argument in favor of compatibilism can be significantly strengthened if we incorporate a notion of modal agency. Since agents are capable of contemplating metaphysical possibilities, the possibility of alternative actions witnessed by metaphysically possible worlds could plausibly be held to indicate that agents can, by virtue of their contemplation of such worlds, ensure that \emph{to them}, choices appear real. In this way, metaphysical contemplation could come to inform our \emph{attitude} towards both our own actions and the actions of others, and free will would become an emergent feature of social reality, physical determinism notwithstanding.

This is clearly a more subtle argument, echoing the arguments made in \cite{strawsonfree}, but backing them up by allowing us to consider relevant attitudes, like resentment and forgiveness, as being grounded in metaphysical facts concerning what it means to contemplate on the metaphysically possible.\footnote{In short, and somewhat simplified, \cite{strawsonfree} argues that since we have attitudes such as these, and since they appear fundamental to how we interact with one another, it does not make sense to hold that they become meaningless if we assume determinism, unless determinism itself is meaningless. On the other hand, since these attitudes seem to presuppose free will, it appears that free will remains meaningful, irrespective of whether or not determinism is true.} Certainly, it might be debated whether or not this is actually a convincing argument that free will and determinism is compatible. However, by taking modal agency into account, we have arrived at a richer vantage point which should be 
helpful to further analysis, both when evaluating existing arguments and also when searching for new ones.\footnote{We mention here the recent argument in favor of compatibilism in \cite{Listfree}, which uses a different terminology, but which we believe can be read as building on a starting point that is, in crucial ways, very similar to ours. This point is elaborated on in the full paper.}

\subsection*{Agency and metaphysical possibility in formal logics}

There is a vast landscape of formal logics that involve agency and possibility, and increasingly, these notions are also considered together, especially in logics for modeling interaction in a multi-agent system, see \cite{IMAS,Benthem}.

In the Humphrey thought experiment, Humphrey knew he lost the election in 1968, but he was still free to contemplate the possibility of a different outcome. By contemplating this possibility, it seems that he is engaging in a form of agency, and while this agency would certainly be related to his actions in the actual world (or at least to his attitudes towards those actions), it would not appear to be a form of agency that we can easily reduce to other forms \hannah{as oppose to ?}.

Towards formal representation of modal agency, it seems clear that we must turn to \emph{multi-modal} logics, where we can study interactions between a modality representing metaphysical possibility, and another, distinct modality, which can be used for talking about agency involving reflection concerning such possibilities. Multi-modal logics is a rich topic which is being studied from many different angles and it attracts much technical interest, see \cite{multimod}. 

In this paper, we will focus on multi-modal logics that are based on a branching time notion of possibility. Such logics have attracted much interest in recent work, both in philosophy and AI, and they are particularly interesting because they have been extended in various ways by adding modal operators specifically directed at modeling agency. We point to \cite{stit,dstit,atl,atle,nctl,stitstart} for a collection of work on such formalisms that is relevant to the points we are making in this paper.

It seems, moreover, that in light of recent work on dispositions, the connection between branching time and metaphysical possibility might be closer than what has hitherto been assumed.\footnote{We point to \cite{MwPw} for a survey of recent work on dispositions and possibility} 

It seems that we may, in light of these new suggestions, consider reinterpretations of the branching time formalisms, viewing transitions between states of the world as resulting from the (possibly counterfactual) \emph{manifestations} of dispositions. 

Then, the temporal dimension can be understood as modeling the \emph{higher order} (counterfactual) manifestation of (possible) dispositions, as explored informally in \cite{dispmod}, where it is suggested that the notion of possibility that arises in this way can be used to provide a plausible actualistic theory of metaphysically possibility proper.

\footnote{We mention that a related development, that also argues for the metaphysical importance of branching time possibility is presented in \cite{realmod}. Here, however, the suggestion is made that branching time possibility is in itself metaphysically basic, in that it gives rise to the \emph{real} notion of metaphysical possibility, which, albeit not as wide as that usually considered, is still wide enough to cover the interesting cases, including those that deserve primary attention in metaphysics.}

In the full paper, we take this point of view further, and we elaborate on various branching time formalisms that have been proposed, focusing on how they incorporate notions of agency, and how they can often be seen to involve several distinct notions of possibility, including a metaphysical notion that should not be conflated to the other ones, especially not in the context of agency.

We go on to suggest that the further philosophical study of different systems and the connections between them has the potential to shed light on a number of different, but related, questions, such as the relationship between free will and determinism \cite{Listfree,strawsonfree}, the workings of higher order dispositions \cite{dispmod}, the applicability of notions involving moral responsibility \cite{frankfurt,mensrea}, the nature of necessity and the question of whether or not dispositional possibility is a distinct form of modality \cite{DM,EaM,Los}, and the distinction between knowing that it is possible to do something, and actually knowing \emph{how} to do it \cite{atlhow,atlhowto}.

We do not aim to provide any definite answers, but make a methodological point: since all of these questions seem to involve the relationship between agency and a metaphysical notion of possibility, more work should be devoted to studying them in this light.

\bibliography{cite}
\bibliographystyle{apalike}
\end{document}
