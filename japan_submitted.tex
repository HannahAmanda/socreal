\documentclass{article}

\usepackage{authblk}
\setlength{\textheight}{655pt} % = 592pt
\setlength{\textwidth}{370pt} % = 345pt
\setlength{\topmargin}{0pt} % = 20pt
\setlength{\voffset}{-35pt} % = 0pt
\setlength{\headsep}{20pt} %= 25pt

\usepackage{fullpage}
\usepackage[curve]{xypic}
 \usepackage{amssymb,amsmath, url,proof,bussproofs} 
\usepackage[usenames,dvipsnames]{color}

\newcommand{\noo}[1]{}
\newcommand{\ptrue}{\sf S}
\newcommand{\pp}{\mathcal P}
\newcommand{\defdby}{\by={def}}  
\newcommand{\naboComp}[1]{{\cal E}(#1)}
\newcommand{\CAP}{\bigcap}
\newcommand{\ungr}[1]{\underline{#1}}
\newcommand{\conc}[1]{} 
\newcommand{\objects}{{\cal O}}
\newcommand{\facts}{{\cal F}}
\newcommand{\states}{{\cal S}}
\newcommand{\powers}{{\cal P}}
\newcommand{\disp}{{\cal D}}
\newcommand{\cause}{{\cal C}}
\newcommand{\dispcomb}{{\pi}}
\newcommand{\irr}{{\cal I}}

\bibliographystyle{apalike}

\hyphenation{}
\newcommand{\rag}[1]{\textbf{\textcolor{magenta}{(R: #1)}}}
\newcommand{\sjur}[1]{\textbf{\textcolor{blue}{(S: #1)}}}
\newcommand{\hannah}[1]{\textbf{\textcolor{OliveGreen}{(H: #1)}}}

\begin{document}

\title{Contemplating counterfactuals: On the connection between agency and metaphysical possibility}
\author[1]{Sjur K. Dyrkolbotn \thanks{s.k.dyrkolbotn@durham.ac.uk}}
\author[2]{Ragnhild H. Jordahl \thanks{ragnhild.jordahl@gmail.com}}
\author[3]{Hannah A. Hansen \thanks{hannaha.hansen@gmail.com}}
\affil[1]{Durham Law School, Durham University, UK}
\affil[2]{Department of Philosophy, University of Bergen, Norway}
\affil[3]{Department of Information Science and Media Studies, University of Bergen, Norway}

\date{}
\maketitle

\section*{Extended Abstract}

We consider the connection between the metaphysics of modality and agency, focusing on how it can be captured in logics for reasoning about multi-agent systems. We argue that philosophical insights can be gained from looking to these formalisms, and that they tend to come with implicit philosophical assumptions that we must consider if we are to understand their broader meaning.

Indeed, social structures that have been designed with the aid of formal tools have increasingly become relevant to social reality, for both real and artificial agents. \footnote{The growing importance of the social web over the last 10-15 years serves as an obvious example of this development.} Hence, philosophical assessments appear especially relevant when formal logics are examined in this context. In addition, philosophy may offer interesting directions to pursue when further developing formal tools.

In the full paper, we first argue that the connection between metaphysical modality and agency needs to be taken into account in order to arrive at a proper understanding of either of these notions. We observe, in particular, that agency appears to feature crucially in important metaphysical arguments concerning possibility, while metaphysical possibility seem to be at play in important arguments concerning agency.

Following up on this, we compare logics of agency with formal approaches to metaphysical theories of modality. We focus attention on branching time temporal logics, particularly variants of \emph{alternating time temporal logic} (ATL) \cite{atl}, and we relate these to the recently proposed \emph{dispositional} account of modality, see \cite{dispmod,MwPw}. The dispositional account makes the connection between possibility, causation and agency clearer at the philosophical level, so providing a formal interpretation of this theory seems like a particularly interesting research challenge.

Proposals have already emerged, giving a formal or semi-formal account of the dispositional theory. We note that these formalisms bear close resemblance to many logics considered in the theory of multi-agent systems and in the philosophy of agency, and we give a brief description and overview of existing work, particularly \cite{powmod,PhDpos}. We go on to argue that the relationships between related formalisms should be considered further. Moreover, by making the connection between metaphysical modality and agency explicit, we hope to shed new light on a number of well-known issues, both from philosophy and the theory of multi-agent systems.

\subsection*{Agency in the metaphysics of possibility}

One of the main controversies in contemporary work on metaphysical modality arises from the tension between the theories of Lewis and Kripke respectively \cite{KripkeNN,KripkeIN,LewisPW,LewisCPB}. Both Lewis and Kripke build on the account given by Leibniz \cite{Theodicy}, who held that something is possible if and only it is true in some possible world, and necessary if and only it is true in all of them.

Lewis' theory relies on an ontology which posits the existence of concretely existing possible worlds, completely separated from our own. On the contrary, Kripke's theory is based on an \emph{actualistic} understanding of possible worlds; what actually exists is taken to be that which is part of our world, and all that is possible must, in principle, originate from this actuality.

It is commonly accepted that a powerful argument can be made against Lewis' theory by considering \emph{identity} and \emph{de re} modal claims, that is, claims about what is possible for a particular existing object, the identity of which we know in the actual world. How can it be, for instance, that something which is possible for \emph{me} is witnessed by the existence of some other world, all the while I myself am part of this one? Lewis answers by saying that what is possible for me is witnessed by something which obtains in some possible world for someone who is not me, but is very much like me, namely my \emph{counterpart} \cite{LewisCPB}. 

This answer is held by many to be an affront to our intuitive understanding of modality. In a famous thought experiment \cite{KripkeNN}, Kripke makes this point by considering the possibility that Humphrey won the 1968 US presidential election. Why exactly would Humphrey care if someone very much like him won the election? Surely, when contemplating the possibility of victory, Humphrey is thinking about \emph{himself}?

Kripke's argument, and the question of identity across possible worlds more generally, seems to owe much of its significance from considerations rooted in agency. Notice, for instance, that modal agency, involving an agent contemplating the possible, is the performative core of the Humphrey thought experiment. More generally, whenever a modal claim becomes pressing in real life, this is invariably due to some agent engaging in modal reflection.\footnote{That is not to say that modal agency subsumes or is constitutive of metaphysical possibility; this would involve excluding many possibilities that are often included in a metaphysical account, such as the possibility of a world with no agents (some may want their metaphysical theory to exclude this, but we prefer to remain agnostic about it). We are not, in particular, suggesting any kind of fictionalism about metaphysical possibilities, and the point we are making is not subsumed by previous work in this vein, as that of \cite{ficr,ficrfix}. While agency should also be considered by such theories, their primary concern is with how possible worlds are to be made sense of, and how they come to be. This is not our topic; our argument is that \emph{regardless} of what possible states of affairs are, it appears that how we \emph{interact} 
with these in our social lives is relevant, also to the formulation of an appropriate metaphysical theory of possibility.} Rather, our point is that quite some relevance is already attributed to it in existing work, and that metaphysical theories therefore need to address modal agency more explicitly. It also seems clear that when doing so, due note should be taken of the fact that thinking about modality is invariably embedded in structures that are present in physical and social reality.

\subsection*{The dispositional account of metaphysical possibility}

On the dispositional account, the possible is determined by dispositions found in the actual world; we remain rooted in this world, and we describe modality as something that is \emph{present} (i.e. not a phenomenon arising simply from the way we tend to use our language for example). To say that something is possible means that there is some actual disposition for which this possibility --- this possible state of affairs --- is its manifestation. The (possible) manifestations can serve to characterize and individuate dispositions, but as dispositions themselves are actual, \emph{they} determine what is metaphysically possible -- what could possibly manifest -- not the other way around. Then we need not rely on possible worlds (real or metaphorical) as a primitive philosophical notion. Possible states of affairs can still be modeled formally as points in a directed graph -- a powerful tool in modal logics -- but according to the dispositional account this does not imply any commitments regarding possible worlds, not even their existence. Rather, possible states of affairs can be \emph{traced back} to their origin in actuality, and while they have rich internal structure, this structure arises from how they could have come about, so that the discourse of possible worlds can remain entirely metaphorical without challenging the reality of metaphysical modalities.\footnote{We point to \cite{MwPw} for a survey of recent work on dispositions and possibility}

It is important to emphasize that dispositions always trace back to properties of objects present in the world here and now. New dispositions do not spontaneously appear along any (counterfactual) future time-lines, and all possibilities result from the possible manifestations of existing dispositions. Still, \emph{higher-order} dispositions might need to be considered, i.e., dispositions that are merely possible and arise from manifestations of dispositions that are always closer -- in a chain of possible manifestations -- to dispositions existing in the actual world, see \cite{dispmod}.

The manifestations of dispositions might or might not come about, and objects tend to have many dispositions that will never materialize. Think of the glass that has the dispositional property of being fragile --- this means that the glass will break if struck with sufficient force, but this disposition to break might very well never become actual. But even if the dispositions are never manifested, the existence of dispositional properties is enough to account for the possibility that the glass \emph{might} break or that it \emph{could have been} broken.

The connection between agency and dispositions can be elucidated by considering the term \emph{powers}. It is used in the philosophy of causation, often as a
synonym for dispositions [Mumford and Anjum, 2011b], but also in the philosophy of
agency, where it has a different, but related, meaning [van Inwagen, 1983]. Roughly
speaking, a power can be seen as a disposition involving agency by way of pointing to
an ability that an agent has to bring about an outcome. In the example above, one
might say of the glass that it is disposed to break, but one might also say of an agent that
he has the power to break it. It seems wrong, however, to say that he is disposed to
do so, simply because he can.

The distinction could be useful for a dispositional theory of possibility. If someone claims "it is possible for me to break the glass", it seems that the disposition of the glass to break if he hits it is no longer a sufficient truthmaker. What if, for instance, we consider a world where this person does not exist, or he is necessarily prevented from hitting the glass for some other reason? In this case, it seems natural to also make reference to his power to hit the glass, not only the dispositional fact that it might break if he does so. 

The interrelated nature of powers and dispositions is further underlined by the observation that mathematically  speaking, the formal frameworks used in \cite{powmod,PhDpos} to study objects and their dispositions are strikingly similar to logics used to study agents and their actions in the theory of multi-agent systems. This, in particular, is the starting point for our technical project, which aims to give an account of the dispositional theory, as well as the connection to agency, by means of multi-agent logics.

\subsection*{Agency and metaphysical possibility in formal logics}

There is a vast landscape of formal logics that involve agency and possibility, and increasingly, these notions are also considered together, especially in logics for modeling interaction in a multi-agent system, see \cite{IMAS,Benthem}.

In the Humphrey thought experiment, Humphrey knew he lost the election in 1968, but he was still free to contemplate the possibility of a different outcome. By contemplating this possibility, it seems that he was engaging in a form of agency, and while this agency was certainly related to his actions in the actual world (or at least to his attitudes towards those actions), this does not appear to be a form of agency that we can easily reduce to other forms. For instance, it does not seem possible to readily account for it in terms of causal decision theory, which considers agency and rational choice in the light of philosophical accounts of causality, see e.g., \cite{joyce}. Indeed, it is too late for that, Humphrey has already lost, so for an account centered on utility-maximizing, his thoughts about winning, in hindsight, appear \emph{irrelevant}. But as anyone who has ever entertained such thoughts knows, this is a gross oversimplification. For \emph{Humphrey} it matters, and it might come to influence his future course of action, particularly whith regards to his \emph{new} goals, and how he will go about trying to achieve them.

Towards a formal representation this, we must turn to \emph{multi-modal} logics, allowing us to study interactions between a modality representing metaphysical possibility, and another, distinct modality, which can be used for talking about agency involving reflection concerning such possibilities.\footnote{Multi-modal logics is a rich topic which is being studied from many different angles and it attracts much technical interest, see \cite{multimod}.} In this paper, we will focus on multi-modal logics that are based on a branching time notion of possibility. Such logics have attracted much interest, both in philosophy and AI, and they are particularly interesting because they have been extended in various ways by adding modal operators specifically directed at modeling agency. We point to \cite{stit,dstit,atl,atle,nctl,stitstart} for a collection of work on such formalisms that is relevant to the points we are making in this paper.

We consider reinterpretations of the branching time formalisms, viewing transitions between states of the world as resulting from the (possibly counterfactual) \emph{manifestations} of dispositions. The temporal dimension can be understood as modeling the \emph{higher order} (counterfactual) manifestation of (possible) dispositions, as explored informally in \cite{dispmod}.\footnote{We mention that a related development, that also argues for the metaphysical importance of branching time possibility is presented in \cite{realmod}. Here, however, the suggestion is made that branching time possibility is in itself metaphysically basic, in that it gives rise to the \emph{real} notion of metaphysical possibility, which, albeit not as wide as that usually considered, is still wide enough to cover the interesting cases, including those that deserve primary attention in metaphysics.}

In the full paper, we take this point of view further, suggesting that further study of such systems and the connections between them has the potential to shed light on a number of different, but related, questions, such as the relationship between free will and determinism \cite{Listfree,strawsonfree}, the workings of higher order dispositions \cite{dispmod}, the applicability of notions involving moral responsibility \cite{frankfurt,mensrea}, the nature of necessity and the question of whether or not dispositional possibility is a distinct form of modality \cite{DM,EaM,Los}, and the distinction between knowing that it is possible to do something, and actually knowing \emph{how} to do it \cite{atlhow,atlhowto}.

The primary aim is to make a methodological point: since all of these questions involve the relationship between agency and metaphysical possibility, more work should be devoted to studying them in this light. By suggesting a formal interpretation of the dispositional theory we hope to make a convincing argument for the soundness of this research project.

\bibliography{cite}
\bibliographystyle{apalike}
\end{document}
